\documentclass[aspectratio=169]{beamer}
\usetheme{metropolis}
\metroset{numbering=fraction}
\usecolortheme{owl}
\usepackage{amsmath,fontenc,amsfonts,mathtools}%
\setbeamertemplate{section in toc}[sections numbered]
\setbeamertemplate{subsection in toc}[subsections numbered]

\makeatletter
\setbeamertemplate{section page}{
	\centering
	\begin{minipage}{22em}
		\raggedright
		\usebeamercolor[fg]{section title}
		\usebeamerfont{section title}
		\thesection.~\insertsectionhead\\[-1ex]
		\usebeamertemplate*{progress bar in section page}
		\par
		\ifx\insertsubsectionhead\@empty\else%
		\usebeamercolor[fg]{subsection title}%
		\usebeamerfont{subsection title}%
		\thesection.\thesubsection.~\insertsubsectionhead
		\fi
	\end{minipage}
	\par
	\vspace{\baselineskip}
}

\makeatother


\usepackage{graphicx}
\usepackage{svg}
%\usepackage{amsmath}
%\usepackage[backend=bibtex,style=chem-acs]{biblatex}
%\addbibresource{careyfoster.bib}
\usepackage{csvsimple,longtable,booktabs}


\title{Enquiry into Iterative schemes}
\subtitle{A Brief Review}
\author{Shashvat Jain}
\date{\today}
\begin{document}
	\maketitle
	\begin{frame}{OUTLINE}
		\tableofcontents
	\end{frame}
	\section{INTRODUCTION}
	\begin{frame}{why not row reduction?}
		Why do we need Iterative methods?\\
		we already have row-reduction,cramer's rule and solution using inverse at our disposal!?\\
		Compuatational complexity of Gauss-elimination is $\mathrm{O(n^3)}$ 
	\end{frame}

	\section{HEURISTICS}
	\begin{frame}{Why is the only }
		\begin{itemize}
			\item \textbf{AIM : } To determine an unknown low resistance using the Carey Foster's bridge.
			\item \textbf{APPARATUS USED : } Carey foster's bridge, unknown low resistance,Resistance box, Battery, jockey, one way key, Galvanometer,small shunt resistance, connecting wires of almost zero resistance.
		\end{itemize}
	\end{frame}

	\subsection{CIRCUIT DIAGRAM}

%	\begin{frame}{CIRCUIT DIAGRAM}
%		\begin{figure}
%			\centering
%			\includegraphics[width=0.7\columnwidth]{Carey_Foster_bridge.png}
%			\caption{The Carey Foster bridge. The thick-edged areas are busbars of almost zero resistance.\cite{dia1}}
%			\label{fig:1}
%		\end{figure}
%	\end{frame}
	

	\subsection{THEORY}
	\begin{frame}{THEORY}
		X is the unknown resistance. P,Q and Y are known resistances of magnitude comparable to that of X, forming the other half of the bridge. The bridge wire EF has a jockey contact D placed along it and is slid until the galvanometer G measures zero. The thick-bordered areas are thick copper busbars of almost zero resistance.\\ 
		The bridge is said to be balanced when no current passes through the galvanometer.\\ 
		\begin{itemize}
			\item
		\end{itemize}
	\end{frame}


	\begin{frame}{FORMULA USED}
		and add 1 to each side:
		\begin{equation}
			{\displaystyle {P \over Q}+1={{X+Y+\sigma (100+\alpha +\beta )} \over {X+\sigma (100-\ell _{2}+\beta )}}}
		\end{equation}
		From (1) and (2) we get:
		\begin{equation*}
			{Y+\sigma (100-\ell _{1}+\beta )=X+\sigma (100-\ell _{2}+\beta )}
		\end{equation*}
		\begin{equation}
			{\implies X= Y+ \sigma (\ell _{2}-\ell _{1})}
		\end{equation}
	\end{frame}
	\begin{frame}{Subtleties of Modification}
		\begin{itemize}
			\item Note that the unknown unwanted resistances $\alpha$ and $\beta$ have no affect on the finally obtained resistance $X$. This reduces the error in the result to a great extent.
			This boosts the sensitivity of the instrument 
			\item Comparing this to a metre bridge, This setup ensures that the components of the circuit are not majorly harmed incase the unknown resistance is very small.
			\item This enables more accurate measurements of smaller resistances.
		\end{itemize}
	\end{frame}
	\subsection{PROCEDURE}
	\begin{frame}{PROCEDURE}
		\textbf{To find the unknown resistance(X)}
		\begin{enumerate}
			\item Setup the circuit as shown in figure \ref{fig:1}
			\item Start by switching on the circuit and sliding the galvanometer jockey ubtil the deflections become very small.
			\item Now remove the shunt resistance and search for null point in the region of minimal deflection.
			\item Once the null point is found, measure EF.
			\item Swap X and Y and repeat the above steps to get $\ell_2$.
			\item Repeat 1,2,3 qnd 4 with different values of Y.
		\end{enumerate}
		\textbf{To find the resistance per unit length ($\sigma$)}
		\begin{enumerate}
			\item Setup the circuit as shown in circuit diagram but now replace X with a wire of zero resistance. 
		\end{enumerate}
	\end{frame}


\section{REFERENCES}
	\begin{frame}[t]
	\frametitle{REFERENCES}
		%\printbibliography[heading=none]
	\end{frame}
\end{document}
